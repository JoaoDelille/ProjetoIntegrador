\section{Analytical Hierarchy Process (AHP)}
\label{ahp}

No processo de seleção do conceito a adotar e desenvolver, foram definidos 4 critérios, com vários subcritérios, com base nos objetivos pretendidos para a aeronave e nas falhas de mercado existentes.

Primeiro, a Segurança, assegurando que a configuração da aeronave é segura para os pacientes e para a equipa médica, em todas as operações médicas ou de voo necessárias. Segundo, o Tempo, isto é, que a aeronave tenha os tempos mais reduzidos possível, importante para responder rapidamente às ocorrências e vital para salvar vidas. Terceiro, o Desempenho, garantindo que a aeronave supera as atuais ambulâncias aéreas ao nível dos tempos e do alcance, sem comprometer o equilíbrio necessário em relação às emissões e ao ruído. Quarto, o Tamanho, isto é, que a aeronave é compacta o suficiente para aterrar tanto nos heliportos das infraestruturas médico-hospitalares como nas zonas das ocorrências, sem desprezar o número de passageiros que a mesma pode transportar.


\subsection{Safety}


Quanto ao critério da Segurança, foram definidos 2 subcritérios. Primeiro, a Segurança em Voo, garantindo que, no geral, é proporcionado conforto ao paciente (por exemplo, a cabine minimizar o ruído vindo das turbinas) durante o transporte. Segundo, a Segurança na Descolagem e Aterragem Verticais, assegurando que o funcionamento da aeronave enquanto o paciente é socorrido não interfere com o bem-estar do paciente e da equipa médica e na preparação do paciente para o embarque e desembarque da aeronave.


\begin{table}[H]
\begin{center}
\caption{Importância relativa dos vários subcritérios do critério da Segurança.}
\begin{tabular}{ |c|c c| }
 \hline
 \textbf{Safety} & \textbf{Flight} & \textbf{VTOL}  \\
\hline
 \textbf{Flight} & 1 & 1/3  \\
 \textbf{VTOL} & 3 & 1   \\
\hline
\end{tabular}
\end{center}
\end{table}

Decidiu-se atribuir uma importância relativa maior ao subcritério da Segurança na Descolagem e Aterragem Verticais, uma vez que a preparação do paciente, o seu embarque e desembarque, são operações a ter em conta, principalmente numa aeronave cujas asas podem interferir com a entrada/saída de passageiros, algo que não acontece com helicópteros.


\subsubsection{Flight Safety}

Ao nível da Segurança em Voo, todos os conceitos oferecem semelhante segurança; contudo, as configurações envolvendo dutos destacam-se ligeiramente por garantem um melhor conforto do paciente e da equipa médica, durante o voo. Assim, as configurações têm as seguintes classificações:


\begin{table}[H]
\begin{center}
\caption{Comparação entre os vários conceitos, para o subcritério da Segurança em Voo.}
\begin{tabular}{ |c|c c c c| }
 \hline
 \textbf{C} & \textbf{1} & \textbf{2} & \textbf{3} & \textbf{4}  \\
\hline
 \textbf{1} & 1 & 1/2 & 1/4 & 1/3 \\
 \textbf{2} & 2 & 1 & 1/3 & 1/3  \\
 \textbf{3} & 4 & 3 & 1 & 1  \\
 \textbf{4} & 3 & 3 & 1 & 1  \\
\hline
\end{tabular}
\end{center}
\end{table}








\subsubsection{VTOL Safety}

Ao nível da Segurança na Descolagem e Aterragem Verticais, assumiu-se que todas as configurações exceto a de asa convencional criavam menos interferências e ofereciam maior segurança no que toca ao embarque e desembarque do paciente, com todas as necessidades médicas envolvidas. Assim, as configurações têm as seguintes classificações:



\begin{table}[H]
\begin{center}
\caption{Comparação entre os vários conceitos, para o subcritério da Segurança na Descolagem e Aterragem Verticais.}
\begin{tabular}{ |c|c c c c| }
 \hline
 \textbf{C} & \textbf{1} & \textbf{2} & \textbf{3} & \textbf{4}  \\
\hline
 \textbf{1} & 1 & 1/7 & 1/7 & 1/6 \\
 \textbf{2} & 7 & 1 & 1/2 & 1/2  \\
 \textbf{3} & 7 & 2 & 1 & 1  \\
 \textbf{4} & 6 & 2 & 1 & 1  \\
\hline
\end{tabular}
\end{center}
\end{table}












\subsection{Time}

Quanto ao critério do Tempo, foram definidos 2 subcritérios. Primeiro, o Tempo de Envio e Resposta, garantindo que a aeronave é rápida a sair da sua base e a chegar ao local da ocorrência, após o alerta. Segundo, o Tempo de Retorno, assegurando que a aeronave regressa rapidamente à sua base e fica disponível para realizar uma nova missão, salvando vidas mas também contribuindo para a redução da frota, e consequentemente, dos custos das empresas prestadoras destes serviços.




\begin{table}[H]
\begin{center}
\caption{Importância relativa dos vários subcritérios do critério do Tempo.}
\begin{tabular}{ |c|c c| }
 \hline
 \textbf{Time} & \textbf{Dispatch / Chute} & \textbf{Return}  \\
\hline
 \textbf{Dispatch / Chute} & 1 & 1/3  \\
 \textbf{Return} & 3 & 1   \\
\hline
\end{tabular}
\end{center}
\end{table}

Decidiu-se atribuir uma importância relativa maior ao subcritério do Tempo de Retorno, uma vez que, como será mencionado novamente de seguida, o Tempo de Envio e Resposta já não pode ser mais otimizado, enquanto que o tempo de retorno pode ser significativamente melhorado em comparação com as atuais ambulâncias aéreas.



\subsubsection{Dispatch \& Chute Time}

Ao nível do Tempo de Envio e Resposta, assumiu-se uma determinada distância para o local da ocorrência. Igualmente, de acordo com a literatura, este tempo é difícil de reduzir, dado que já se encontra praticamente otimizado, uma vez que não depende apenas da aeronave mas, principalmente, da preparação da equipa e material médicos. Assim, as configurações têm classificações próximas:

\begin{table}[H]
\begin{center}
\caption{Comparação entre os vários conceitos, para o subcritério do Tempo de Resposta.}
\begin{tabular}{ |c|c c c c| }
 \hline
 \textbf{C} & \textbf{1} & \textbf{2} & \textbf{3} & \textbf{4}  \\
\hline
 \textbf{1} & 1 & 1 & 1/2 & 1/2 \\
 \textbf{2} & 1 & 1 & 1/2 & 1/2  \\
 \textbf{3} & 2 & 2 & 1 & 1  \\
 \textbf{4} & 2 & 2 & 1 & 1  \\
\hline
\end{tabular}
\end{center}
\end{table}




\subsubsection{Return Time}


Ao nível do Tempo de Retorno, os quatro conceitos foram pensados para terem o mesmo tempo de retorno para o seu alcance. Por exemplo, embora a configuração convencional tenha uma velocidade cruzeiro máxima significativamente inferior à configuração com dutos em asa em \textit{tandem}, tem igualmente alcance muito menor, pelo que tem praticamente o mesmo tempo de retorno. Assim, todas as configurações têm igual classificação:


\begin{table}[H]
\begin{center}
\caption{Comparação entre os vários conceitos, para o subcritério do Tempo de Regresso.}
\begin{tabular}{ |c|c c c c| }
 \hline
 \textbf{C} & \textbf{1} & \textbf{2} & \textbf{3} & \textbf{4}  \\
\hline
 \textbf{1} & 1 & 1 & 1 & 1 \\
 \textbf{2} & 1 & 1 & 1 & 1  \\
 \textbf{3} & 1 & 1 & 1 & 1  \\
 \textbf{4} & 1 & 1 & 1 & 1  \\
\hline
\end{tabular}
\end{center}
\end{table}






\subsection{Performance}

Quanto ao critério do Desempenho, foram definidos 4 subcritérios. Primeiro, a Velocidade Cruzeiro Máxima, garantindo que a aeronave é rápida o suficiente, reduzindo a duração do transporte e do retorno à base. Segundo, as Emissões, assegurando que a aeronave se encontra dentro daquilo que é a transição energética do setor aeronáutico (\textit{green aviation}). Terceiro, o Alcance, garantindo uma maior cobertura de serviços de ambulâncias aéreas, contribuindo para uma melhor operacionalização da frota e dos recursos humanos por parte das empresas prestadoras destes serviços. Quarto, o Ruído, de modo a não perturbar os moradores onde a ocorrência se dá nem prejudicar a saúde da equipa médica e do paciente.





\begin{table}[H]
\begin{center}
\caption{Importância relativa dos vários subcritérios do critério do Desempenho.}
\begin{tabular}{ |c|c c c c| }
 \hline
 \textbf{Performance} & \textbf{Max Cruise Speed} & \textbf{Emissions} & \textbf{Range} & \textbf{Noise}  \\
\hline
 \textbf{Max Cruise Speed} & 1 & 9 & 2 & 6 \\
 \textbf{Emissions} & 1/9 & 1 & 1/9 & 1/2  \\
 \textbf{Range} & 1/2 & 9 & 1 & 6  \\
 \textbf{Noise} & 1/6 & 2 & 1/6 & 1  \\
\hline
\end{tabular}
\end{center}
\end{table}


Quanto à importância relativa dos vários subcritérios, decidiu-se destacar os subcritérios da Velocidade Cruzeiro Máxima e do Alcance, por serem absolutamente centrais para cumprir os objetivos definidos para este projeto de ambulância aérea e para corrigir as falhas de mercado identificadas. Também pesou nesta decisão o facto dos quatro conceitos terem um nível de desempenho semelhante ao nível das emissões e por serem todas aeronaves híbridas, inserindo-se num contexto de \textit{green aviation}, mas também porque, de acordo com \textbf{citar estudo}, a população tolera níveis elevados de ruído provenientes das ambulâncias aéreas, uma vez que se tratam de emergências, estando vidas em causa.




\subsubsection{Maximum Cruise Speed}

Ao nível da Velocidade Cruzeiro Máxima, as configurações com recurso a rotores inclinados exibem um melhor resultado, atingindo maiores velocidades. Note-se que, embora este subcritério seja dos mais importantes, até porque contribui ativamente para a redução dos tempos, o processo da AHP implica um consenso entre este e outros critérios e subcritérios, como o alcance, o tamanho e o ruído, de modo a cumprir todos os objetivos pretendidos para a aeronave e a corrigir falhas de mercado existentes. Assim, as classificações são as seguintes:



\begin{table}[H]
\begin{center}
\caption{Comparação entre os vários conceitos, para o subcritério da Velocidade Cruzeiro Máxima.}
\begin{tabular}{ |c|c c c c| }
 \hline
 \textbf{C} & \textbf{1} & \textbf{2} & \textbf{3} & \textbf{4}  \\
\hline
 \textbf{1} & 1 & 2 & 3 & 9 \\
 \textbf{2} & 1/2 & 1 & 3 & 5  \\
 \textbf{3} & 1/3 & 1/3 & 1 & 3  \\
 \textbf{4} & 1/9 & 1/5 & 1/3 & 1  \\
\hline
\end{tabular}
\end{center}
\end{table}









\subsubsection{Emissions}

Ao nível das Emissões, os quatro conceitos, como já foi referido anteriormente, apresentam um desempenho semelhante a este nível, tratando-se de aeronaves híbridas, inserindo-se, por isso, em \textit{green aviation}. Assim, todas as configurações têm igual classificação:



\begin{table}[H]
\begin{center}
\caption{Comparação entre os vários conceitos, para o subcritério das Emissões.}
\begin{tabular}{ |c|c c c c| }
 \hline
 \textbf{C} & \textbf{1} & \textbf{2} & \textbf{3} & \textbf{4}  \\
\hline
 \textbf{1} & 1 & 1 & 1 & 1 \\
 \textbf{2} & 1 & 1 & 1 & 1  \\
 \textbf{3} & 1 & 1 & 1 & 1  \\
 \textbf{4} & 1 & 1 & 1 & 1  \\
\hline
\end{tabular}
\end{center}
\end{table}





\subsubsection{Range}

Ao nível do Alcance, as configurações com recurso a rotores apresentam maior alcance. Note-se que os motores são elétricos, com as baterias a serem alimentados por geradores a combustível (gás), tratando-se, por isso, de aeronaves híbridas. Assim, as classificações são as seguintes:



\begin{table}[H]
\begin{center}
\caption{Comparação entre os vários conceitos, para o subcritério do Alcance.}
\begin{tabular}{ |c|c c c c| }
 \hline
 \textbf{C} & \textbf{1} & \textbf{2} & \textbf{3} & \textbf{4}  \\
\hline
 \textbf{1} & 1 & 2 & 6 & 8 \\
 \textbf{2} & 1/2 & 1 & 5 & 7  \\
 \textbf{3} & 1/6 & 1/5 & 1 & 2  \\
 \textbf{4} & 1/8 & 1/7 & 1/2 & 1  \\
\hline
\end{tabular}
\end{center}
\end{table}






\subsubsection{Noise}

Ao nível do Ruído, embora este seja tolerável pela população, é sempre positivo quando a aeronave em desenvolvimento apresenta baixos níveis de ruído, o que acontece no caso de configurações com dutos. Assim, as classificações são as seguintes:


\begin{table}[H]
\begin{center}
\caption{Comparação entre os vários conceitos, para o subcritério do Ruído.}
\begin{tabular}{ |c|c c c c| }
 \hline
 \textbf{C} & \textbf{1} & \textbf{2} & \textbf{3} & \textbf{4}  \\
\hline
 \textbf{1} & 1 & 1 & 1/5 & 1/5 \\
 \textbf{2} & 1 & 1 & 1/5 & 1/5  \\
 \textbf{3} & 5 & 5 & 1 & 1  \\
 \textbf{4} & 5 & 5 & 1 & 1  \\
\hline
\end{tabular}
\end{center}
\end{table}





\subsection{Size}

Quanto ao critério do Tamanho, foram definidos 2 subcritérios. Primeiro, o Número de Passageiros, garantindo que a aeronave consegue transportar o número típico de passageiros num voo de assistência médica. Segundo, a Envergadura e o Comprimento da Fuselagem, assegurando que a aeronave é suficientemente compacta para aterrar em heliportos (que, usualmente, nos hospitais, têm dimensões 14x14m) e nas zonas das ocorrências, que podem ser muito distintas, desde zonas urbanas até florestas, montanhas e locais remotos.


\begin{table}[H]
\begin{center}
\caption{Importância relativa dos vários subcritérios do critério do Tamanho.}
\begin{tabular}{ |c|c c| }
 \hline
 \textbf{Size} & \textbf{No. Passengers} & \textbf{Span + Length}  \\
\hline
 \textbf{No. Passengers} & 1 & 1/7  \\
 \textbf{Span + Length} & 7 & 1   \\
\hline
\end{tabular}
\end{center}
\end{table}

Decidiu-se atribuir uma importância relativa muito maior ao subcritério da Envergadura e do Comprimento da Fuselagem, uma vez que são diferentes nos quatro conceitos, enquanto que o Número de Passageiros é o mesmo.




\subsubsection{Number of Passengers}

Ao nível do Número de Passageiros, os quatro conceitos foram pensados para o mesmo número de passageiros, isto é, o paciente mais 5 membros da equipa médica que, de acordo com a literatura disponível, varia entre 3 a 7 membros, sendo que, usualmente, apenas se recorre a mais que 5 membros para voos de longa distância, que não são o objetivo desta ambulância aérea. Assim, todas as configurações têm igual classificação:

\begin{table}[H]
\begin{center}
\caption{Comparação entre os vários conceitos, para o subcritério do Número de Passageiros.}
\begin{tabular}{ |c|c c c c| }
 \hline
 \textbf{C} & \textbf{1} & \textbf{2} & \textbf{3} & \textbf{4}  \\
\hline
 \textbf{1} & 1 & 1 & 1 & 1 \\
 \textbf{2} & 1 & 1 & 1 & 1  \\
 \textbf{3} & 1 & 1 & 1 & 1  \\
 \textbf{4} & 1 & 1 & 1 & 1  \\
\hline
\end{tabular}
\end{center}
\end{table}





\subsubsection{Span \& Length}

Ao nível da Envergadura e do Comprimento da Fuselagem, os conceitos foram classificados conforme o quão compactos eram. Note-se que o uso de asa em \textit{tandem} é bastante útil para diminuir a envergadura da asa principal, quando estamos perante aeronaves de grande porte. Abaixo, apresenta-se a classificação:

\begin{table}[H]
\begin{center}
\caption{Comparação entre os vários conceitos, para o subcritério da Envergadura e do Comprimento da Fuselagem.}
\begin{tabular}{ |c|c c c c| }
 \hline
 \textbf{C} & \textbf{1} & \textbf{2} & \textbf{3} & \textbf{4}  \\
\hline
 \textbf{1} & 1 & 1/4 & 1/4 & 1/6 \\
 \textbf{2} & 4 & 1 & 1 & 1/2  \\
 \textbf{3} & 4 & 1 & 1 & 1/2  \\
 \textbf{4} & 6 & 2 & 2 & 1  \\
\hline
\end{tabular}
\end{center}
\end{table}



\subsection{Decision}

Assim, com base na ponderação dos critérios e subcritérios e nas características de cada configuração, obteve-se, com recurso ao \textit{MATLAB}, os resultados deste processo, abaixo indicados:


\begin{table}[H]
\begin{center}
\caption{Resultados da \textit{Analytical Hierarchy Process} (AHP), obtidos com recurso ao \textit{MATLAB}.}
\begin{tabular}{ l l  }
 \textbf{Concepts} &    \\
 \hline
 Tilt rotor on conventional wing: &  0.210297 \\
 Tilt rotor on tandem wing: & 0.244391 \\
 Duct on conventional wing: & 0.258704  \\
 \textbf{Duct on tandem wing:} & \textbf{0.286608}   \\
%\hline
\end{tabular}
\end{center}
\end{table}


Neste sentido, com base no resultado da AHP, decidiu-se então adotar a configuração de dutos em asa em \textit{tandem}. No entanto, ao desenvolver este projeto, percebeu-se que o recurso a dutos ainda é muito pioneiro na indústria aeronáutica existindo, por isso, pouca informação e literatura e poucos exemplos disponíveis, de forma a poder avançar com este projeto utilizando as ferramentas disponíveis. Assim, decidiu-se adotar a configuração melhor classificada que não recorria a dutos, ou seja, a configuração de rotores inclinados em asa em \textit{tandem}.
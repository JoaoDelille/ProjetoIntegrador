\section{Noise \& Pollutant Emissions}
\label{noise}
Atualmente, a aviação é um setor em expansão e, juntamente com este, verifica-se um preocupante crescimento de emissões de poluentes e de ruído com consequências prejudiciais tanto para o ser humano, como para os ecossistemas. Assim, desde a fase de design conceptual de uma aeronave, é fundamental ter tal em consideração.

Quanto à aeronave em causa, por se tratar de uma ambulância aérea, vamos procurar reduzir as suas emissões ao máximo, sem, contudo, comprometer a vida dos doentes que transporta.

\section{Emissões de Poluentes}
Alterações climáticas, aquecimento global e qualidade do ar são alguns dos mais alarmantes problemas que têm adquirido relevância nas últimas décadas. É de notar que as aeronaves são a única fonte antropogénica de poluentes na atmosfera e, como tal, é essencial procurar reduzir as suas emissões. %se for necessário pode tirar-se este parágrafo





\section{Emissões de Ruído}
Ruído pode ser definido como som não desejado pelo ouvinte. Algumas das suas consequências são:

\begin{itemize}
    \item Indução de stress e transtorno do sono;
    \item Produz irritação e perturba a concentração;
    \item Causa um decréscimo no desempenho humano;
    \item Pode causar doença física e perda permanente de audição (caso se trate de uma exposição repetida a som intenso).
\end{itemize}

Desta forma, o controlo e redução dos níveis de ruído tornaram-se uma necessidade na nossa sociedade tecnológica devido à aumento contínuo das fontes de ruído.